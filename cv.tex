%%%%%%%%%%%%%%%%%%%%%%%%%%%%%%%%%%%%%%%%%%%%%%%%%%%%%%%%%%%%%%%%%%%%%%
% LaTeX Template: Curriculum Vitae
%
% Source: http://www.howtotex.com/
% Feel free to distribute this template, but please keep the
% referal to HowToTeX.com.
% Date: July 2011
%Version for spanish users, by dgarhdez
% 
%%%%%%%%%%%%%%%%%%%%%%%%%%%%%%%%%%%%%%%%%%%%%%%%%%%%%%%%%%%%%%%%%%%%%%
% How to use writeLaTeX: 
%
% You edit the source code here on the left, and the preview on the
% right shows you the result within a few seconds.
%
% Bookmark this page and share the URL with your co-authors. They can
% edit at the same time!
%
% You can upload figures, bibliographies, custom classes and
% styles using the files menu.
%
% If you're new to LaTeX, the wikibook is a great place to start:
% http://en.wikibooks.org/wiki/LaTeX
%
%%%%%%%%%%%%%%%%%%%%%%%%%%%%%%%%%%%%%%%%%%%%%%%%%%%%%%%%%%%%%%%%%%%%%%
\documentclass[paper=a4,fontsize=11pt]{scrartcl} % KOMA-article class
							
\usepackage[english]{babel}
\usepackage[utf8x]{inputenc}
\usepackage[protrusion=true,expansion=true]{microtype}
\usepackage{amsmath,amsfonts,amsthm}     % Math packages
\usepackage{graphicx}                    % Enable pdflatex
\usepackage[svgnames]{xcolor}            % Colors by their 'svgnames'
\usepackage{geometry}
	\textheight=700px                    % Saving trees ;-)
\usepackage{url}

\frenchspacing              % Better looking spacings after periods
\pagestyle{empty}           % No pagenumbers/headers/footers

%%% Custom sectioning (sectsty package)
%%% ------------------------------------------------------------
\usepackage{sectsty}

\sectionfont{%			            % Change font of \section command
	\usefont{OT1}{phv}{b}{n}%		% bch-b-n: CharterBT-Bold font
	\sectionrule{0pt}{0pt}{-5pt}{1pt}}

%%% Macros
%%% ------------------------------------------------------------
\newlength{\spacebox}
\settowidth{\spacebox}{8888888888}			% Box to align text
\newcommand{\sepspace}{\vspace*{1em}}		% Vertical space macro

\newcommand{\MyName}[1]{ % Name
		\Huge \usefont{OT1}{phv}{b}{n} \hfill #1
		\par \normalsize \normalfont}
		
\newcommand{\MySlogan}[1]{ % Slogan (optional)
		\large \usefont{OT1}{phv}{m}{n}\hfill \textit{#1}
		\par \normalsize \normalfont}

\newcommand{\NewPart}[1]{\section*{\uppercase{#1}}}

\newcommand{\PersonalEntry}[2]{
		\noindent\hangindent=2em\hangafter=0 % Indentation
		\parbox{\spacebox}{        % Box to align text
		\textit{#1}}		       % Entry name (birth, address, etc.)
		\hspace{1.5em} #2 \par}    % Entry value

\newcommand{\SkillsEntry}[2]{      % Same as \PersonalEntry
		\noindent\hangindent=2em\hangafter=0 % Indentation
		\parbox{\spacebox}{        % Box to align text
		\textit{#1}}			   % Entry name (birth, address, etc.)
		\hspace{1.5em} #2 \par}    % Entry value	
		
\newcommand{\EducationEntry}[4]{
		\noindent \textbf{#1} \hfill      % Study
		\colorbox{White}{%
			\parbox{5cm}{%
			\hfill\color{Black}#2}} \par  % Duration
		\noindent \textit{#3} \par        % School
		\noindent\hangindent=2em\hangafter=0 \small #4 % Description
		\normalsize \par}

\newcommand{\WorkEntry}[4]{				  % Same as \EducationEntry
		\noindent \textbf{#1} \hfill      % Jobname
		\noindent\colorbox{Black}{\color{White}#2} \par  % Duration
		\noindent \textit{#3} \par              % Company
		\noindent\hangindent=2em\hangafter=0 \small #4 % Description
		\normalsize \par}

%%% Begin Document
%%% ------------------------------------------------------------
\begin{document}
% you can upload a photo and include it here...
%\begin{wrapfigure}{l}{0.5\textwidth}
%	\vspace*{-2em}
%		\includegraphics[width=0.15\textwidth]{photo}
%\end{wrapfigure}

\MyName{Luis Felipe González Rivas}
\MySlogan{Licenciado en Matemáticas}

\sepspace

%%% Personal details
%%% ------------------------------------------------------------
\NewPart{Datos personales}{}

%\PersonalEntry{Nacimiento}{3 de febrero, 1993}
%\PersonalEntry{Address}{111 First St, New York}
\PersonalEntry{Teléfono}{8121-099-669}
\PersonalEntry{E--Mail}{\url{luisgzzrivas@gmail.com}}
\PersonalEntry{Sitio}{\url{luisgrivas.com}}
\PersonalEntry{GitHub}{\url{https://github.com/luisgrivas}}

%%% Education
%%% ------------------------------------------------------------
\NewPart{Educación}{}

\EducationEntry{Licenciatura en Matemáticas}{Agosto 2010 - Diciembre 2014}{Universidad Autónoma de Nuevo León}{
\begin{itemize}
%\item{Especialidad }
\item{Estadística, probabilidad, métodos numéricos, optimización, investigación de operaciones, álgebra lineal.}
%\item{Proyecto de fin de carrera: \emph{título del proyecto}}
\end{itemize}
}

\sepspace

\EducationEntry{Maestría en Sistemas Inteligentes}{2015-2018 (por terminar)}{Instituto Tecnológico y de Estudios Superiores de Monterrey}{
\begin{itemize}
%\item{Especialidad }
\item{Inteligencia artificial, machine learning, estadística, matemáticas aplicadas, sistemas de incertidumbre, sistemas de búsqueda y razonamiento, diseño de algoritmos.}
\item{Proyecto de tesis: \emph{Aplicación de Redes Neuronales Convolucionales para clasificación de señales EEG}}
\end{itemize}
}
%\EducationEntry{Cursos y certificados}{
%\begin{itemize}
%item 
%\end{itemize}


%%% Work experience
%%% ------------------------------------------------------------
\NewPart{Experiencia laboral}{}

\EducationEntry{Data Scientist}{Mayo 2017 - Noviembre 2017}{ITESM}{
Procesamiento de bases de datos e impementación de modelos matemáticos para predecir eventos relevantes para la institución.}
\sepspace

\EducationEntry{Data Scientist}{Noviembre 2017 - Mayo 2018}{Vector Casa de Bolsa}{\begin{itemize}
\item{Creación de modelos matemáticos y de inteligencia artificial para eficientizar procesos internos}
\item{Análisis de datos e implementación de modelos predictivos para generar oportunidades de negocio}
\item{Creación de modelos de datos mediante algoritmos de data mining para la toma de decisiones}
%\item{Coordinación de proyectos con enfoque de ciencia de datos}
\end{itemize}}

\sepspace

\EducationEntry{Catedrático de matemáticas}{Enero 2018 - Junio 2018}{Facultad de Ciencias Físico Matemáticas - UANL}{Impartir curso de Teoría de Grupos  para estudiantes de la licenciatura en matemáticas}

%%% Skills
%%% ------------------------------------------------------------
\NewPart{Habilidades personales}{}

\SkillsEntry{Idiomas}{Español (nativo)}
\SkillsEntry{}{Inglés (avanzado)}
%\SkillsEntry{}{Francés (nivel)}

\SkillsEntry{Software}{Python, R, SQL, C++, MATLAB, LISP, PROLOG,  LaTeX}
%\SkillsEntry{Soft skills}{Trabajo en equipo}
%\SkillsEntry{}{Flexibilidad y adaptación ante cambios en el proyecto}
%\SkillsEntry{}{Análisis de datos con enfoque de en solución de problemas}
%\SkillsEntry{}{Análisis y resolución de problemas}
%\SkillsEntry{}{}




%%% References
%%% ------------------------------------------------------------
%\NewPart{References}{}
%Available upon request
\end{document}

