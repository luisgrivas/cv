\documentclass{resume} % Use the custom resume.cls style

%\usepackage[left=0.4 in,top=0.4in,right=0.4 in,bottom=0.4in]{geometry} % Document margins


\usepackage[left=0.75in,top=0.6in,right=0.75in,bottom=0.6in]{geometry} % Document margins

\name{Luis González Rivas} % Your name
% You can merge both of these into a single line, if you do not have a website.
\address{luisgrivas.com \\ luisgzzrivas@gmail.com }%

\begin{document}
%----------------------------------------------------------------------------------------
%	OBJECTIVE
%----------------------------------------------------------------------------------------

%\begin{rSection}{Objetivos}
%{ Cre}
%\end{rSection}

%----------------------------------------------------------------------------------------
%	WORK EXPERIENCE SECTION
%----------------------------------------------------------------------------------------
\begin{rSection}{EXPERIENCIA}

%------------------------------------------------
\begin{rSubsection}{Talisis}{Diciembre 2022 - Actual}{Científico de Datos}{Monterrey, NL}
    \item Creación de modelos de IA Generativa adaptados a la generación de contenido educativo.
    \item Mejora de sistemas de recomendación para Talisis.com.
\end{rSubsection}

%------------------------------------------------
\begin{rSubsection}{Vector Casa de Bolsa}{Noviembre 2018 - Octubre 2020}{Gerente de Vector Analytics}{Monterrey, NL}
\item Dirigir y desarrollar proyectos de ciencia de datos para las múltiples áreas de negocio.
\item Colaboración en el desarrollo de CRM con integración de modelos de \textit{machine learning.}
\end{rSubsection}

%------------------------------------------------
\begin{rSubsection}{Vector Casa de Bolsa}{Diciembre 2017 - Noviembre 2018}{Científico de Datos}{Monterrey, NL}

\item Procesamiento y unificación de múltiples bases de datos en un \textit{DataLake.}
\item Desarrollo de \textit{Vector Analytics API} para la producción modelos de \textit{machine learning}.
\item Creación de modelos de predicción de abandono de clientes, sistemas de recomendación de productos de la casa de bolsa y modelos de \textit{pricing} de divisas.
\item Implementación de modelos de \textit{people analytics} para capital humano.
\item Producción de \textit{dashboards} utilizando información procesada en \textit{DataLake.}
\end{rSubsection}

%------------------------------------------------
{\bf Facultad de Ciencias Físico Matemáticas - UANL} \hfill {Enero 2018 - Diciembre 2020}\\{\em Profesor universitario de matemáticas} \hfill {\em Monterrey, NL}

%------------------------------------------------
\begin{rSubsection}{ Intituto Tecnólogico y de Estudios Superiores de Monterrey}{Mayo 2017 - Diciembre 2017}{Científico de Datos}{Monterrey, NL}
\item Procesamiento datos institucionales relacionados al cuerpo docente, estudiantes y exalumnos para posterior análisis estadístico.
\item Generación de \textit{insights} mediante técnicas de minería de datos para directivos y responsables de distintas áreas de la institución.
\item Creación de tableros mediante \textit{Tableau} para análisis de datos.

\end{rSubsection}
\end{rSection} 
%----------------------------------------------------------------------------------------
%	EDUCATION SECTION
%----------------------------------------------------------------------------------------

\begin{rSection}{Educación}

{\bf Universidad Autónoma de Nuevo León} \hfill {Agosto 2010 - Diciembre 2014}\\
Licenciatura en Matemáticas

{\bf Intituto Tecnólogico y de Estudios Superiores de Monterrey} \hfill {Agosto 2015 - Diciembre 2017} \\
Maestría en Ciencias con Especialidad en Sistemas Inteligentes.\\Completado 33 créditos.

{\bf Universidad Nacional Autónoma de México } \hfill {Enero 2021 - Julio 2023 (esperada)}\\
Maestría en Ciencias Matemáticas
\end{rSection}


%----------------------------------------------------------------------------------------
% TECHINICAL STRENGTHS	
%----------------------------------------------------------------------------------------

\begin{rSection}{Habilidades}

\begin{tabular}{ @{} >{\bfseries}l @{\hspace{6ex}} l }
%Idiomas: Inglés\\
Habilidades &  Procesamiento de datos, aprendizaje máquina, análisis estadístico \\
Lenguajes de programación &  Python, R,  C++, SQL, ArangoDB, MATLAB\\
Librerías & Keras, Scikit-Learn, Pandas, NumPy, Flask, , dplyr, ggplot
\end{tabular}

\end{rSection}



%\begin{rSection}{PROJECTS}
%\vspace{-1.25em}
%\item \textbf{Hiring Search Tool.} {Built a tool to search for Hiring Managers and Recruiters by using ReactJS, NodeJS, Firebase and boolean queries. Over 25000 people have used it so far, with 5000+ queries being saved and shared, and search results even better than LinkedIn! \href{https://hiring-search.careerflow.ai/}{(Try it here)}}
%\item \textbf{Short Project Title.} {Build a project that does something and had quantified success using A, B, and C. This project's description spans two lines and also won an award.}
%\item \textbf{Short Project Title.} {Build a project that does something and had quantified success using A, B, and C. This project's description spans two lines and also won an award.}
%\end{rSection} 

%----------------------------------------------------------------------------------------
%\begin{rSection}{Extra-Curricular Activities} 
%\begin{itemize}
 %   \item 	Actively write \href{https://www.faangpath.com/blog/}{blog posts} and social media posts (\href{https://www.tiktok.com/@faangpath}{TikTok}, \href{https://www.instagram.com/faangpath/?hl=en}{Instagram}) viewed by over 20K+ job seekers per week to help people with best practices to land their dream jobs. 
  %  \item	Sample bullet point.
%\end{itemize}
%\end{rSection}

%----------------------------------------------------------------------------------------
%\begin{rSection}{Leadership} 
%\begin{itemize}
 %   \item Admin for the \href{https://discord.com/invite/WWbjEaZ}{FAANGPath Discord community} with over 6000+ job seekers and industry mentors. Actively involved in facilitating online events, career conversations, and more alongside other admins and a team of volunteer moderators! 
%\end{itemize}
%\end{rSection}


\end{document}
