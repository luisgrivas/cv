%%%%%%%%%%%%%%%%%%%%%%%%%%%%%%%%%%%%%%%%%
% Medium Length Professional CV
% LaTeX Template
% Version 2.0 (8/5/13)
%
% This template has been downloaded from:
% http://www.LaTeXTemplates.com
%
% Original author:
% Trey Hunner (http://www.treyhunner.com/)
%
% Important note:
% This template requires the resume.cls file to be in the same directory as the
% .tex file. The resume.cls file provides the resume style used for structuring the
% document.
%
%%%%%%%%%%%%%%%%%%%%%%%%%%%%%%%%%%%%%%%%%

%----------------------------------------------------------------------------------------
%	PACKAGES AND OTHER DOCUMENT CONFIGURATIONS
%----------------------------------------------------------------------------------------

\documentclass{resume} % Use the custom resume.cls style

\usepackage[left=0.75in,top=0.6in,right=0.75in,bottom=0.6in]{geometry} % Document margins

\name{Luis González Rivas} % Your name
\address{8121099669 \\ luisgzzrivas@gmail.com} % Your phone number and email

\begin{document}

%----------------------------------------------------------------------------------------
%	EDUCATION SECTION
%----------------------------------------------------------------------------------------

\begin{rSection}{Educación}

{\bf Universidad Autónoma de Nuevo León} \hfill {\em 2010-2014} \\ 
Licenciatura en matemáticas \\
Asesor académico en materias de matemáticas \smallskip 

{\bf Instituto Tecnológico y de Estudios Superiores de Monterrey} \hfill {\em 2015-2017 (sin terminar)} \\
Maestría en Sistemas Inteligentes\\
Tesis: Aplicación de Redes Neuronales Convolucionales para clasificación de señales EEG. \smallskip 

{\bf Universidad Nacional Autónoma de México} \hfill {\em 2021-2022 (en curso)} \\
Maestría en Ciencias Matemáticas
\end{rSection}

%----------------------------------------------------------------------------------------
%	WORK EXPERIENCE SECTION
%----------------------------------------------------------------------------------------

\begin{rSection}{Experiencia laboral}

\begin{rSubsection}{Instituto Tecnológico y de Estudios Superiores de Monterrey}{Mayo 2017 - Diciembre 2017}{Científico de Datos}{Monterrey, México}
\item Responsable de procesar datos institucionales relacionados al cuerpo docente, estudiantes y EXATECs. Además, utilizando técnicas de minería de datos y estadística, una de mis principales funciones era generar \textit{insights} para directivos y responsables de distintas áreas de la institución.
\end{rSubsection}

%------------------------------------------------

\begin{rSubsection}{Vector Casa de Bolsa}{Diciembre 2017 - Noviembre 2018}{Científico de Datos}{Monterrey, México}
\item Responsable de procesar y unificar las bases de datos de distintas áreas de negocio en un DataLake. A partir de esto, una de mis principales contribuciones fue crear modelos de predicción de abandono de clientes, sistemas de recomendación de productos de la casa de bolsa y modelos de \textit{pricing} de divisas. Para la producción de modelos, desarrollé un API que traducía los resultados de los modelos en datos aprovechables para las distintas plataformas de la empresa. Finalmente, fui responsable del desarrollo y producción de dashboards que ofrecían insights a distintas áreas de negocio de la empresa.
\end{rSubsection}


%------------------------------------------------

\begin{rSubsection}{Vector Casa de Bolsa}{Noviembre 2018 - Octubre 2020}{Gerente de Vector Analytics}{Monterrey, México}
\item Responsable del área de ciencias de datos de la empresa (Vector Analytics). Dentro de mis funciones, era proponer, desarrollar y entregar nuevos proyectos de ciencias de datos para las distintas áreas de negocio. Una de mis principales contribuciones fue dirigir y participar en el desarrollo del CRM que utilizaba modelos de aprendizaje máquina para ofrecer un mejor servicio a los clientes. 
\end{rSubsection}

%------------------------------------------------

\begin{rSubsection}{Facultad de Ciencias Físico Matemáticas - UANL}{Enero 2018 - Diciembre 2020}{Profesor universitario de matemáticas}{Monterrey, México}
\item Responsable de impartir las asignaturas de álgebra abstracta, análisis real, teoría de la medida y cálculo vectorial.
\item Instructor del taller de ciencias de datos para estudiantes de matemáticas.
\end{rSubsection}
%------------------------------------------------

\end{rSection}

%----------------------------------------------------------------------------------------
%	TECHNICAL STRENGTHS SECTION
%----------------------------------------------------------------------------------------

\begin{rSection}{Habilidades}

\begin{tabular}{ @{} >{\bfseries}l @{\hspace{6ex}} l }
Idiomas: Inglés\\
Habilidades: Minería de datos, modelado matemático, diseño de algoritmos, aprendizaje máquina. \\
Lenguajes de programación: Python, R, SQL, C++, MATLAB, Lisp, Prolog, LaTeX.
\end{tabular}

\end{rSection}

%\begin{rSection}{Awards and Recognitions}

%\begin{tabular}{ @{} >{\bfseries}l @{\hspace{6ex}} l }
%Techfugees Hackathon (2018) &  Our team (Elevate) won Second Place. \\
%Sydney University & Advertised Bursary (2018),NSW Health Grant (2017), Robert Maple Brown (2013)\\
%Macquarie University (2013) & Overall Excellence in Science\\
%\end{tabular}

%\end{rSection}

%\begin{rSection}{Section Name}

%Section content\ldots

%\end{rSection}

%----------------------------------------------------------------------------------------

\end{document}
